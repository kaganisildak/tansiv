\documentclass[a4paper,11pt]{article}

\usepackage{fullpage}
\usepackage[utf8]{inputenc}
\usepackage[OT1]{fontenc}
\usepackage{xspace}
\usepackage{url}
\usepackage{pdfswitch}
\usepackage{comment}
\usepackage{paralist}
\usepackage[compact]{titlesec}
\renewcommand{\rmdefault}{phv}
\setlength{\parskip}{0.3em}

\begin{document}

\title{Time-accurate Network Simulation\\ Interconnecting QEMU VMs}

\author{}

\date{}
\maketitle

\vspace{-1.2cm}

\pagestyle{empty}
\thispagestyle{empty}

\noindent \textbf{Executive summary:} The proposed work aims at
designing an evaluation environment for distributed infrastructures
where the instances of the real application are executed in full
QEMU-based virtual machines, interconnected by a network simulator.
Starting from an exisiting prototype, the work should enable the study
of unmodified real infrastructures such as the \textit{Storm} Event
Processing Infrastructure and the \textit{Ceph} Distributed Storage.

\noindent\textbf{Advisors:}\vspace{-.6\baselineskip}
\begin{itemize}
\item Martin Quinson (ENS-Rennes, IRISA, team Myriads)
  \texttt{Martin.Quinson@ens-rennes.fr}\vspace{-.5\baselineskip} 
\item Benjamin Camus (CNRS, IRISA, team Myriads)
  \texttt{Benjamin.Camus@irisa.fr}\vspace{-.5\baselineskip} 
\item Louis Rilling (DGA, team Myriads) \texttt{Louis.Rilling@irisa.fr} 
\end{itemize}\vspace{-.5\baselineskip}

\noindent \textbf{Level:} Master.
% -level Internship} of 4 to 6 months (with potential follow-up in a
% PhD thesis).\\ 

\noindent \textbf{Key skills:} Deep understanding of OSes, Networks
and VMs; System Programming on Linux.


\medskip
\subsection*{Context}

Distributed systems such as Peer-to-Peer systems, Internet of Things
and Cloud Computing, benefit of an ever increasing popularity
nowadays.  Distributed applications (such as decentralized data
sharing solutions, games, scientific applications, high-traffic web
applications) are executed routinely on these systems.

By nature, the resulting environments and applications are extremely
complex and dynamic because they aggregate thousands of heterogeneous
and dynamic elements. This make these systems very challenging to
study, test, and evaluate.  Purely theoretical studies rely on
assumptions that are at best simplistic and often unrealistic. Most of
the studies are thus done through experiments, often on dedicated
facilities.
%
But the recent evolution of the target systems in size, dynamicity and
complexity makes it difficult to even test the infrastructures in a
reliable and reproducible manner. An appealing alternative is to rely
on simulation.

%\medskip

SimGrid (developed by M. Quinson in an international collaboration) is
a toolkit providing core functionality for the simulation of
distributed applications in heterogeneous distributed environments.
% 195 workshops, 79 revues, 11 book chapters, 34 thesis
Over the years, SimGrid has emerged as one of the key scientific
instrument in this domain. It grounded the experiments of over 30 PhDs
works, 90 journal articles and book chapters as well as 200 conference
papers, and is cited by over 500 other articles in the literature. Its
key features are its sound performance models (enabling accurate
performance prediction in non-trivial scenarios) as well as its
efficiency and scalability.
% , and its ability to formally assess the correctness of distributed
% algorithms through the exhaustive exploration of the possible
% executions.

The performances of MPI applications can directly be evaluated with
SimGrid, as the standard is reimplemented on top of the simulator. For
other real applications, users have to extract the applicative logic
and rewrite it using the SimGrid interfaces. Several application
authors did so to test and tune their application within the simulator
but this work remains tedious and error-prone. The overall goal of
this internship is to leverage the predictive power of SimGrid on
unmodified, non-trivial distributed applications.

\subsection*{Description}

%\subsection*{Detailed work plan}

The proposed work consists in three main steps:
\begin{itemize}
\item Modifying QEMU's user mode network emulation (based on SLIRP) to make the VM
communicate with the network simulator instead of the real network. This is the
core of the technical contribution of the internship.
%This step will be already partly achieved at the beginning of the internship.
\item Building a minimal required network environment to run the applications, that is typically a DNS and possibly file and mail servers. The building blocks could use parts of QEMU's user mode network emulation as well as real servers in virtual machines. Building this environment should be automated.
\item Validating the network environment on top of the network simulator by experimenting with real applications. We especially propose to experiment with ShareLatex and Ceph to gradually stress the implemented mechanisms and show their completeness.
\end{itemize}
The experiments will leverage the Grid'5000 testbed and associated tools like EnosLib developed in the Myriads team.

%\subsection*{Skills required}
%
%In addition to the skills that can reasonably be expected from Master-level
%students, the applicant should have a \textbf{very strong} knowledge of system
%programming in C, and of Linux and other modern Unix-based Operating Systems.

\subsection*{Bibliography}
\begin{compactitem}
%\item SimGrid : \url{http://simgrid.gforge.inria.fr/}
%\item Tutor: \url{http://www.loria.fr/~quinson/}
\item H. Casanova, A. Giersch, A. Legrand, M.  Quinson and F. Suter.
  \textit{Versatile, Scalable, and Accurate Simulation of Distributed
    Applications and Platforms}, Journal of Parallel and Distributed Computing
  74(10), 2014. \url{http://hal.inria.fr/hal-01017319}.
\item M. Guthmuller. \textit{Dynamic formal verification of temporal
    properties over real distributed applications}. PhD thesis,
  2015. \url{https://tel.archives-ouvertes.fr/tel-01231868}
\item M. Quinson. \textit{Computational Science of Computer Systems.}
  HdR thesis, 2013. \url{https://tel.archives-ouvertes.fr/tel-00927316}
\item H. Song, X. Liu, D. Jakobsen, R. Bhagwan, X. Zhang, K. Taura,
  and A. Chien. \textit{MicroGrid: a scientific tool for modeling
    computational grids.} In SuperComputing Conf. 2000.
%\item Brian White, Jay Lepreau, Leigh Stoller, Robert Ricci, Shashi
%  Guruprasad, Mac Newbold, Mike Hibler, Chad Barb, and Abhijeet
%  Joglekar. An integrated experimental environment for distributed
%  systems and networks. SIGOPS Oper. Syst. Rev., 36(SI) :255–270,
%  December 2002.
% \item Benjamin Quétier, Vincent Neri, and Franck Cappello. Scalability
%   comparison of four host virtualization tools. Journal of Grid
%   Computing, 5(1) :83–98, 2007.
\item J. Mirkovic, T. Benzel, T.  Faber, R. Braden, J. Wroclawski and
  S. Schwab. \textit{The DETER Project: Advancing the Science of Cyber
    Security Experimentation and Test.}  In Proceedings of the IEEE
  Technologies for Homeland Security Conference 2010 (HST'10).
\item \url{https://www.grid5000.fr/}
\item \url{https://gitlab.inria.fr/discovery/enoslib}
\end{compactitem}

\end{document}

%  LocalWords:  AlGorille Distem SimGrid Simterpose
