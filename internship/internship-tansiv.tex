\documentclass[a4paper,11pt]{article}

\usepackage{fullpage}
\usepackage[utf8]{inputenc}
\usepackage[OT1]{fontenc}
\usepackage{xspace}
\usepackage{url}
\usepackage{pdfswitch}
\usepackage{comment}
\usepackage{paralist}
\usepackage[compact]{titlesec}
\renewcommand{\rmdefault}{phv}
\setlength{\parskip}{0.3em}

\begin{document}

\title{Time-accurate Network Simulation\\ Interconnecting QEMU VMs}

\author{}

\date{}
\maketitle

\vspace{-1.2cm}

\pagestyle{empty}
\thispagestyle{empty}

\noindent \textbf{Executive summary:} The proposed work aims at
designing an evaluation environment for distributed infrastructures
where the instances of the real application are executed in full
QEMU-based virtual machines, interconnected by a network simulator.
Starting from an exisiting prototype, the work should enable the study
of unmodified real infrastructures such as the \textit{Storm} Event
Processing Infrastructure and the \textit{Ceph} Distributed Storage.

\noindent\textbf{Advisors:}\vspace{-.6\baselineskip}
\begin{itemize}
\item Martin Quinson (ENS-Rennes, IRISA, team Myriads)
  \texttt{Martin.Quinson@ens-rennes.fr}\vspace{-.5\baselineskip} 
\item Benjamin Camus (CNRS, IRISA, team Myriads)
  \texttt{Benjamin.Camus@irisa.fr}\vspace{-.5\baselineskip} 
\item Louis Rilling (DGA, team Myriads) \texttt{Louis.Rilling@irisa.fr} 
\end{itemize}\vspace{-.5\baselineskip}

\noindent \textbf{Level:} Master.
% -level Internship} of 4 to 6 months (with potential follow-up in a
% PhD thesis).\\ 

\noindent \textbf{Key skills:} Deep understanding of OSes, Networks
and VMs; System Programming on Linux.


\medskip
\subsection*{Context}

Distributed systems such as Peer-to-Peer systems, Internet of Things
and Cloud Computing, benefit of an ever increasing popularity
nowadays.  Distributed applications (such as decentralized data
sharing solutions, games, scientific applications, high-traffic web
applications) are executed routinely on these systems.

By nature, the resulting environments and applications are extremely
complex and dynamic because they aggregate thousands of heterogeneous
and dynamic elements. This make these systems very challenging to
study, test, and evaluate.  Purely theoretical studies rely on
assumptions that are at best simplistic and often unrealistic. Most of
the studies are thus done through experiments, often on dedicated
facilities.
%
But the recent evolution of the target systems in size, dynamicity and
complexity makes it difficult to even test the infrastructures in a
reliable and reproducible manner. An appealing alternative is to rely
on simulation.

%\medskip

SimGrid (developed by M. Quinson in an international collaboration) is
a toolkit providing core functionality for the simulation of
distributed applications in heterogeneous distributed environments.
% 195 workshops, 79 revues, 11 book chapters, 34 thesis
Over the years, SimGrid has emerged as one of the key scientific
instrument in this domain. It grounded the experiments of over 30 PhDs
works, 90 journal articles and book chapters as well as 200 conference
papers, and is cited by over 500 other articles in the literature. Its
key features are its sound performance models (enabling accurate
performance prediction in non-trivial scenarios) as well as its
efficiency and scalability.
% , and its ability to formally assess the correctness of distributed
% algorithms through the exhaustive exploration of the possible
% executions.

The performances of MPI applications can directly be evaluated with
SimGrid, as the standard is reimplemented on top of the simulator. For
other real applications, users have to extract the applicative logic
and rewrite it using the SimGrid interfaces. Several application
authors did so to test and tune their application within the simulator
but this work remains tedious and error-prone. The overall goal of
this internship is to leverage the predictive power of SimGrid on
unmodified, non-trivial distributed applications.

\subsection*{Description}
This proposal aims at extending over the current situation in two
directions. 

First, we want to enable the study of arbitrary applications through a
specifically tailored virtualization mechanism. More precisely, all
system calls would have to be intercepted.  The communications would
be mediated according to the results computed by the simulator while
the computations would only be benchmarked to reinject their timings
into the simulator. 

Several proof of concepts were developed by previous interns, but many
problems remain to be solved, both on the theoretical and practical
sides. The applicant is expected to contribute a framework to mediate
any actions of arbitrary distributed applications. The developed proof
of concept is necessary to the practical evaluation of the
contribution, and it will be technically reinforced by an engineer
afterward. The SimGrid models will have to be reviewed and possibly
adapted to improve the simulation accuracy in this new settings.

The second work direction aims at the formal verification of real
applications. If the performance simulation only requires an
virtualization mechanism that can observe the application, the formal
verification needs to save and restore the application state, and also
to introspect this state.
%
To the best of our knowledge, no such virtualization layer enables
these operations on arbitrary applications. The solutions existing in
SimGrid are a very strong basis, but they will have to be extended to
allow to the study of separate processes directly at the system layer.

\subsection*{Detailed work plan}
After a refresh of our literature review, the first year will be
devoted to the technical framework. The existing proof of concepts
will be extended and combined to enable the emulation of simplistic
distributed applications. Evaluating solutions that are intended to
assess distributed applications can be challenging \textit{per se},
and the applicant will also have to setup an appropriate testing
framework with relevant test applications. This framework itself will
be exercised through the assessment and improvement of the performance
models.

During the second year, the student will turn his/her interception
framework dedicated to the performance simulation into a full
virtualization framework enabling the formal verification of arbitrary
applications. Several existing features of the SimGrid dynamic
verification tool will be adapted for the verification of separate
arbitrary applications: save \& restore, incremental checkpointing and
state introspection.

During the last year of this work, the student will apply his/her work
on complex applications under realistic workloads, and write the
manuscript. The ultimate goal is to assess modern distributed
infrastructures, such as the Ceph distributed storage solution, the
Storm event processing system or the Samba networked disk server.



% \item Conduct a throughout bibliographical study of the field. The
%   goal is to gain a better understanding of the system parts that
%   Simterpose should emulate (CPU, communication, DNS, threads, etc),
%   and the possible approaches for each aspect.
% \item Implement the selected approaches. This development is necessary
%   to the practical evaluation of the contribution. The goal is to only
%   develop a proof of concept, that will be technically reinforced by
%   an engineer afterward.
% \item Evaluate the feasibility of the contribution. An evaluation
%   framework should be designed, with the selection beforehand of the
%   applications and workload. The goal is to study which approaches are
%   \textit{possible} to lightly virtualize distributed
%   applications. The emulation performance (that is, the overhead of
%   the virtualization) should also be assessed. 
% \item Evaluate the correctness of our models in this context. The
%   simulation results should be compared to measurements on real
%   platforms such as Grid'5000, and (if time permits) the SimGrid
%   models should be modified if they prove unadapted to this new
%   context.


\subsection*{Skills required}

In addition to the skills that can reasonably be expected from Master-level
students, the applicant should have a \textbf{very strong} knowledge of system
programming in C, and of Linux and other modern Unix-based Operating Systems.

\subsection*{Bibliography}
\begin{compactitem}
%\item SimGrid : \url{http://simgrid.gforge.inria.fr/}
%\item Tutor: \url{http://www.loria.fr/~quinson/}
\item H. Casanova, A. Giersch, A. Legrand, M.  Quinson and F. Suter.
  \textit{Versatile, Scalable, and Accurate Simulation of Distributed
    Applications and Platforms}, Journal of Parallel and Distributed Computing
  74(10), 2014. \url{http://hal.inria.fr/hal-01017319}.
\item M. Guthmuller. \textit{Dynamic formal verification of temporal
    properties over real distributed applications}. PhD thesis,
  2015. \url{https://tel.archives-ouvertes.fr/tel-01231868}
\item M. Quinson. \textit{Computational Science of Computer Systems.}
  HdR thesis, 2013. \url{https://tel.archives-ouvertes.fr/tel-00927316}
\item H. Song, X. Liu, D. Jakobsen, R. Bhagwan, X. Zhang, K. Taura,
  and A. Chien. \textit{MicroGrid: a scientific tool for modeling
    computational grids.} In SuperComputing Conf. 2000.
%\item Brian White, Jay Lepreau, Leigh Stoller, Robert Ricci, Shashi
%  Guruprasad, Mac Newbold, Mike Hibler, Chad Barb, and Abhijeet
%  Joglekar. An integrated experimental environment for distributed
%  systems and networks. SIGOPS Oper. Syst. Rev., 36(SI) :255–270,
%  December 2002.
% \item Benjamin Quétier, Vincent Neri, and Franck Cappello. Scalability
%   comparison of four host virtualization tools. Journal of Grid
%   Computing, 5(1) :83–98, 2007.
\item J. Mirkovic, T. Benzel, T.  Faber, R. Braden, J. Wroclawski and
  S. Schwab. \textit{The DETER Project: Advancing the Science of Cyber
    Security Experimentation and Test.}  In Proceedings of the IEEE
  Technologies for Homeland Security Conference 2010 (HST'10).
\end{compactitem}

\end{document}

%  LocalWords:  AlGorille Distem SimGrid Simterpose
