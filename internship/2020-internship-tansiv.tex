\documentclass[a4paper,11pt]{article}

\usepackage[margin=25mm]{geometry}
\usepackage[utf8]{inputenc}
\usepackage[OT1]{fontenc}
\usepackage{xspace}
\usepackage{url}
\usepackage{pdfswitch}
\usepackage{comment}
\usepackage{paralist}
\usepackage[compact]{titlesec}
\renewcommand{\rmdefault}{phv}
% \setlength{\parskip}{0.3em}
%\setlength{\itemsep}{0em}

\begin{document}

\title{\vspace{-1.2cm}Time-accurate Network Simulation Interconnecting VMs \\ toward stealth analysis}
\author{}
\date{}

\maketitle

\vspace{-1.2cm}

\pagestyle{empty}
\thispagestyle{empty}

\noindent \textbf{Executive summary:} The proposed work aims at designing an evaluation environment
for distributed infrastructures where the instances of the real application are executed in full
QEMU-based virtual machines, interconnected by the SimGrid network simulator. Starting from a working
prototype, the proposed work will enable the study of malicious applications.
%such as the
%\textit{DPDK} framework, or even the stealthy analysis of malicious distributed applications.

\noindent\textbf{Advisors:}\vspace{-.6\baselineskip}
\begin{itemize}
\item Martin Quinson (ENS-Rennes, IRISA, team Myriads)
  \texttt{Martin.Quinson@ens-rennes.fr}\vspace{-.5\baselineskip} 
%\item Benjamin Camus (CNRS, IRISA, team Myriads)
%  \texttt{Benjamin.Camus@irisa.fr}\vspace{-.5\baselineskip} 
\item Louis Rilling (DGA, team Myriads) \texttt{Louis.Rilling@irisa.fr} \vspace{-.5\baselineskip} 
\item Matthieu Simonin (Inria Rennes) \texttt{Matthieu.Simonin@inria.fr}
\end{itemize}\vspace{-.3\baselineskip}

\noindent\textbf{Team:} %\vspace{-.2\baselineskip}
Myriads. \textbf{Laboratory:} IRISA, Rennes (head: Guillaume Gravier -- \url{guig@irisa.fr}).
%\noindent \textbf{Level:} Master.
% -level Internship} of 4 to 6 months (with potential follow-up in a
% PhD thesis).\\ 

\noindent \textbf{Key skills:} Deep understanding of OSes, Networks and VMs; System Programming on
Linux.


\medskip
\subsection*{Context and Description}
\vspace{-.3\baselineskip}

By nature, distributed applications are challenging to analyze and debug because of their size,
complexity and dynamicity. These characteristics make it very difficult to actually test the systems
in a reliable and reproducible manner. Simulation constitutes an appealing alternative toward
convenient experiments, but often either requires to reimplement the target application using a
specific interface, or mandates intrusive inspection techniques.

QEMU is a whole system emulator that can be used as a virtual machine to run the differing nodes of
a distributed infrastructure on a single machine. The Panda project builds upon QEMU for the reverse
engineering and live analysis of arbitrary systems.
%
SimGrid is a simulator of distributed applications in heterogeneous distributed environments. Its
key features are its sound performance models (enabling accurate performance prediction in
non-trivial scenarios) as well as its ability to run directly legacy applications written with the
MPI standard.


This internship aims at leveraging the predictive power of SimGrid on unmodified, non-trivial
distributed applications executed within modified QEMU-based virtual machines. The ultimate goal is
to enable to study codes that detect and evade any analysis. To this end, we want to
extend the stealthiness of existing analysis frameworks by completely simulating the network.

\smallskip

\subsection*{Detailed Work Plan}
\vspace{-.3\baselineskip}

The intern will be provided with a working prototype that can run simple applications between QEMU
VMs through SimGrid, opening many research directions. The intern is expected to choose and complete
two to three work leads from the four ones provided here: \vspace{-.3\baselineskip}

\begin{itemize}
\item \textit{Framework validation on real applications.} The intern will validate the prototype on
  several real distributed applications of increasing complexity, to gradually stress the
  implemented mechanisms. The proposed applications are ShareLaTeX, Ceph and DPDK.
\vspace{-.5\baselineskip} %
\item \textit{Performance evaluation to understand the tool practical usability.} QEMU is currently
  used in full emulation mode, that comes with a high performance penalty. The intern will quantify
  this performance loss for several applications used out of the box. \vspace{-.5\baselineskip} %
\item \textit{Overall approach portability beyond the current implementation.} We want to explore
  how our work can be generalized. We want to remove the need of full emulation (that hinders
  performance) to allow hardware-assisted virtualization. Providing sufficient control over the time in
  this case will require new techniques, as the existing ones heavily depend on the full emulation.
  %
  Similarly, we want to explore how our work could be generalized and adapted to KVM and Xen.
\vspace{-.5\baselineskip} %
\item \textit{Evaluating the framework intrusiveness on sandboxes' analysis tools.} Finally, we want
  to evaluate the interference induced by our module once integrated to open-source analysis
  frameworks such as Panda or Drakvuf. In particular, the strict control of time achieved by our
  module should not hinder the sandbox' analysis power.
\end{itemize}
%The experiments will leverage the Grid'5000 testbed and associated
%tools like EnosStack that are developed in part in the Myriads team.


\subsection*{Bibliography}
\begin{compactitem}
\item H. Casanova, A. Giersch, A. Legrand, M.  Quinson and F. Suter.
  \textit{Versatile, Scalable, and Accurate Simulation of Distributed
    Applications and Platforms}, Journal of Parallel and Distributed Computing
  74(10), 2014. \url{http://hal.inria.fr/hal-01017319}.
\item B. Dolan-Gavitt, J. Hodosh, P. Hulin, T. Leek, R. Whelan. \textit{Repeatable Reverse
    Engineering with PANDA}. 5th Program Protection and Reverse Engineering Workshop, Los Angeles,
  California, December 2015.  \url{https://apps.dtic.mil/sti/pdfs/AD1034415.pdf}
\item T. Lengyel, S. Maresca, B. Payne, G.  Webster, S.  Vogl and A.  Kiayias.
  \textit{Scalability, Fidelity and Stealth in the DRAKVUF Dynamic Malware Analysis System}.
  Proceedings of the 30th Annual Computer Security Applications Conference, 2014.
\item J. Mirkovic, T. Benzel, T.  Faber, R. Braden, J. Wroclawski and
  S. Schwab. \textit{The DETER Project: Advancing the Science of Cyber
    Security Experimentation and Test.}  In Proceedings of the IEEE
  Technologies for Homeland Security Conference 2010 (HST'10).
\item N. Miramirkhani, M. Priya Appini, N. Nikiforakis,
  M. Polychronakis. \textit{Spotless Sandboxes: Evading Malware
    Analysis Systems using Wear-and-Tear Artifacts.}  In Proceedings
  of the 2017 IEEE Symposium on Security and Privacy.
  \url{https://securitee.org/files/wearntear-oakland2017.pdf}
\item H. Tazaki, F. Urbani, E. Mancini, M. Lacage, D. Camara,
  T. Turletti, and W. Dabbous, \textit{Direct Code Execution:
    Revisiting Library OS Architecture for Reproducible Network
    Experiments}. In the 9th International Conference on emerging
  Networking EXperiments and Technologies
  (CoNEXT'13). \url{https://hal.inria.fr/hal-00880870}
\end{compactitem}

\end{document}

%  LocalWords:  AlGorille Distem SimGrid Simterpose
